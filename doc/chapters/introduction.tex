\section{\S About These Notes}
The primary purpose of writing these notes is to organize my knowledge and passion for mathematics while challenging my understanding and capabilities in this field. As an engineering student, My First aim to practice and improve my  writing skills through this endeavor.

I am not a professional mathematician and barely consider myself an engineer, so advanced readers might find errors in my explanations. I encourage readers to remain alert and consult the numerous references I will provide to support my explanations and proofs. My secondary aim is to share my knowledge and passion with others by explaining fundamental mathematical concepts in the simplest terms possible, minimizing the difficulties typically encountered when learning math.

Mathematics is inherently challenging due to its abstract nature. Additionally, in our modern era, finding the time and focus necessary for deep mathematical thinking is increasingly difficult. I recommend that the reader take their time and be patient in order to fully grasp these notes.

\section{\S What is Mathematics?}
In modern society, mathematics is considered an abstract science. To clarify this concept, we can use the analogy of a zoologist. A zoologist travels the world to study animals—concrete entities—by observing their behaviors and interactions within their environments. In contrast, a mathematician studies another type of entity: abstract entities. These entities are termed 'abstract' because you cannot physically point to something and say, 'Here is the number 4 or 5,' as you do can with tangible entities like animals. Similarly, abstract entities such as justice, goodness, or love cannot be physically pinpointed
\footnote{For further clarification on this point, I recommend watching the YouTube video titled "What is Mathematics?" published on January 18, 2020 by channel \textit{Matemáticas en Nuevo León}.}.

Mathematics involves the study of numbers, their interactions with each other and their environment, and their various interpretations. This exploration is facilitated by its foundational elements: logic and intuition, analysis and construction, generality and individuality. These elements are essential for engaging with and understanding these abstract entities. It is important to take note of these foundational elements, as they will be referenced throughout these notes.

From my perspective, the proposed definition of mathematics in the context of these notes seems adequate in its coverage and efficacy. However, it is imperative to reflect on the intricate nature of this field. In this regard, it is essential to consider the very definition of "science," a term that continues to generate debates and questioning in contemporary academic and scientific discourse. The inherent complexity in defining "science" reflects the multifaceted nature of the scientific enterprise, encompassing not only the acquisition and development of knowledge but also epistemological, methodological, and philosophical aspects. Therefore, while the proposed definition of mathematics here may be suitable for the purposes of these notes, it is crucial to recognize the breadth and depth of the discussions surrounding both mathematics and science in general.
\section{\S Why is Mathematics Important?}
Abstract sciences such as mathematics are often undervalued. This may be because their practical relevance is not immediately apparent to the average person. For example, understanding that the sum of two even numbers is also an even number may seem trivial to someone who is not a scientist. Consequently, people tend to value the practical applications of mathematics more than the abstract principles.

This situation highlights a significant issue in how mathematics is currently taught. The focus is often on technical processes and formulas rather than on understanding the underlying concepts. As A. N. Whitehead pointed out learning formulas and techniques without grasping their deeper meaning leads to superficial knowledge
\footnote{For a more in-depth understanding of A. N. Whitehead's perspective, refer to the first chapter, "The Abstract Nature of Mathematics," in his book \textit{An Introduction to Mathematics}.}.

Mathematics, however, is fundamentally important. Human social development is closely linked to scientific advancement, and science aims to objectively analyze and understand natural processes, patterns, and phenomena. This involves identifying patterns, naming them, describing them, and assigning attributes. This systematic approach leads to knowledge and potentially to universal laws.

Consider the example of an apple. Many people might describe an apple based on its taste, smell, and texture—subjective sensations that vary from person to person-. In contrast, abstract knowledge provided by science does not rely on sensory experience and is therefore objective. This is where mathematical abstraction helps a lot, not only because it is abstract but also because it is consistent. If we state that 5 is lesser than 10 and greater than 4, that will be true for all of us. In the case of science, it would describe the apple in terms of the position and motions of its molecules, thus ultimately reducing it to numbers—again, mathematics which is consistent. An ancient Greek philosopher, Pythagoras, believed that the arche or essence of this universe is indeed the number
\footnote{Pythagoras believed that things are as they are due to their form, which can be measured and quantified. For a more detailed analysis, see the final pages of the chapter "Matter and Form" in William K. C. Guthrie's book.}.

Thus, we must appreciate the significance of mathematics. Understanding the abstract principles underlying mathematics is crucial for grasping the complexities of the world around us. Creating knowledge and advancing science will likely help move society towards something better.

Therefore, if our intention is to live better in this world or universe, shouldn’t we strive to understand it at least minimally?

\section{\S The Origins of  Mathematics}
Now, I would like to provide some context. I believe that studying history to understand how these mathematical ideas originated from ancient times is crucial to avoid taking many things for granted. Understanding the evolution of thought is important for sharpening and achieving higher levels of comprehension thoughts.

The history of mathematics constitutes a captivating journey through time, spanning millennia of human development and intellectual discoveries. Its origins trace back to \textbf{\textit{Prehistory}}, where our ancestors cultivated rudimentary concepts of counting and measurement to fulfill fundamental life necessities such as hunting, gathering, and constructing shelters. These initial notions of quantity and space laid the groundwork for the subsequent evolution of mathematical thought.

The earliest written records of mathematics originate from ancient civilizations such as the \textbf{\textit{Sumerians}} (5500 BCE to 1800 BCE), who flourished in \textbf{\textit{Mesopotamia}}. The Sumerians developed a numerical system based on 60, which influenced the division of the circle into 360 degrees and the division of time into hours, minutes, and seconds. This numerical foundation also led to the development of more sophisticated systems of measurement and accounting records.

The \textbf{\textit{Babylonians}} (1894 BCE to 539 BCE), successors to the Sumerians, also made significant contributions to mathematics. They created multiplication and division tables and developed methods for solving quadratic and cubic equations. Their mathematical knowledge was applied in fields such as astronomy and construction, demonstrating an advanced understanding of concepts such as proportion and geometry.

The ancient \textbf{\textit{Egyptians}} (3100 BCE to 332 BCE) possessed a profound mathematical acumen, which they adeptly employed in the construction of monumental structures like the pyramids and in land surveying during the annual inundations of the Nile. They pioneered a base-ten numerical system and were among the earliest civilizations to employ fractions, facilitating the resolution of practical challenges pertaining to food allocation and commercial transactions.

However, it was in the \textbf{\textit{Ancient Greek}} civilization where mathematics took a significant leap towards abstraction and formalization. Figures such as \textbf{\textit{Pythagoras, Euclid, and Archimedes}} laid the groundwork for Western mathematics. Pythagoras established the importance of numerical and geometrical relationships, including the famous Pythagorean theorem. Euclid authored \textbf{\textit{Elements}} a seminal work that systematized \textbf{\textit{Euclidean geometry}} and set a standard for mathematical proof. \textbf{\textit{Archimedes}} made significant contributions in areas such as calculus, geometry, and physics, demonstrating the utility of mathematics in understanding natural phenomena and solving practical problems.

Following the decline of the \textbf{\textit{Roman Empire and the Middle Ages}}, the \textbf{\textit{European Renaissance}} (the 14th century to the 17th century) brought about a resurgence of interest in ancient Greek and Roman wisdom, sparking a new period of mathematical flourishing. Mathematicians such as Leonardo da Vinci, Johannes Kepler, and Galileo Galilei made significant contributions in fields such as geometry, astronomy, and physics, utilizing mathematics as a tool to comprehend the universe and its underlying laws.

The \textbf{\textit{Modern Period}}, spanning from the 17th to the 19th century, witnessed revolutionary advancements in mathematics. \textbf{\textit{Isaac Newton and Gottfried Leibniz}} developed calculus, a set of mathematical techniques that revolutionized the study of change and motion, laying the foundations for modern physics. Furthermore, new branches of mathematics emerged, such as number theory, probability theory, and non-Euclidean geometry, further expanding the scope and applicability of this discipline.

In conclusion, the history of mathematics stands as a testament to human ingenuity and its capacity to explore and comprehend the world around us. From its humble origins in prehistory to its development as a highly abstract and formalized discipline, mathematics has played a pivotal role in advancing science, technology, and human culture at large. Its study and application remain crucial in the modern world, underscoring its relevance and power to transform our understanding of the universe and our ability to address complex problems\footnote{
  Recommend to read a little bit the article called \textit{mathematics} from \textit{Encyclopedia Britannica}
  \href{https://www.britannica.com/science/mathematics}{link}.
}.


















